\documentclass{article}
\usepackage[greek,english]{babel}
\usepackage{amsmath}
\usepackage{algorithm}
\usepackage{algorithmicx}
\usepackage{algpseudocode}
\usepackage{fullpage}

\begin{document}
\author{}
\date{}

\textgreek{Μέλη Ομάδας: }
\begin{itemize}
\item \textgreek{Ιγγλεζάκης Αναστάσιος - 1115202300054}
\item \textgreek{Σκότης Αθανάσιος - 1115202300188}
\end{itemize}

\section*{Exercise 1}
\section*{\textgreek{Περιγραφή Αλγορίθμου: }}
\begin{enumerate}
    \item \textgreek{Αρχικοποίηση του πίνακα αποστάσεων:}
    \[
    \text{dist}[i][j] =
    \begin{cases}
    w(i, j), & \textgreek{αν υπάρχει ακμή } i \rightarrow j \\
    \infty, & \textgreek{διαφορετικά}
    \end{cases}
    \]
    \item \textgreek{Για κάθε κόμβο $k \in V$, και για κάθε ζεύγος κόμβων $i, j \in V$, βρίσκουμε τη συντομότερη διαδρομή μεταξύ $i$ και $j$}:
    \[
    \text{dist}[i][j] = \min(\text{dist}[i][j], \text{dist}[i][k] + \text{dist}[k][j])
    \]
    \item \textgreek{Ο συντομότερος κύκλος δίνεται από:}
    \[
    \min_{v \in V} \text{dist}[v][v]
    \]
    \textgreek{Αν} $\text{dist}[v][v] = \infty$ \textgreek{για όλα τα $v$, τότε το γράφημα δεν περιέχει κύκλο.}
\end{enumerate}

\section*{\textgreek{Ψευδοκώδικας:}}

\begin{algorithm}[H]
\caption{\textgreek{Συντομότερος Κατευθυνόμενος Κύκλος}}
\begin{algorithmic}[1]
\Statex \textgreek{Είσοδος: Γράφος $G = (V, A)$ με θετικά βάρη $w(u, v)$}
\Statex \textgreek{Έξοδος: Μήκος του συντομότερου κατευθυνόμενου κύκλου ή $None$ αν δεν υπάρχει κύκλος}

\State \textgreek{Αρχικοποίησε τον πίνακα} \texttt{dist[$n$][$n$]} \textgreek{ως εξής}:
\ForAll{$i, j \in V$}
    \If{$i = j$}
        \State \texttt{dist[$i$][$j$]} $\gets 0$
    \ElsIf{$(i, j) \in A$}
        \State \texttt{dist[$i$][$j$]} $\gets w(i, j)$
    \Else
        \State \texttt{dist[$i$][$j$]} $\gets \infty$
    \EndIf
\EndFor

\For{$k = 1$ to $n$}
    \For{$i = 1$ to $n$}
        \For{$j = 1$ to $n$}
            \State \texttt{dist[$i$][$j$]} $\gets \min(\texttt{dist[$i$][$j$]}, \texttt{dist[$i$][$k$]} + \texttt{dist[$k$][$j$]})$
        \EndFor
    \EndFor
\EndFor

\State \texttt{minCycle} $\gets \infty$
\For{$v = 1$ to $n$}
    \State \texttt{minCycle} $\gets \min(\texttt{minCycle}, \texttt{dist[$v$][$v$]})$
\EndFor

\If{\texttt{minCycle = $\infty$}}
    \State \Return \texttt{None}
\Else
    \State \Return \texttt{minCycle}
\EndIf

\end{algorithmic}
\end{algorithm}

\section*{\textgreek{Ανάλυση Πολυπλοκότητας: }}

\textgreek{Ο αλγόριθμος αποτελείται από τρεις εμφωλευμένους βρόγχους επανάληψης. Έστω \( n = |V| \)}

\subsection*{Main Loop}

\textgreek{Υπολογίζουμε: }
\[
\text{dist}[i][j] = \min(\text{dist}[i][j], \text{dist}[i][k] + \text{dist}[k][j])
\]
\textgreek{για κάθε } \( i, j, k \in \{1, 2, \dots, n\} \).

\subsection*{\textgreek{Συνολικός αριθμός πράξεων τριπλού βρόγχου:} }

\textgreek{Αθροίζουμε ως προς όλους τους συνδυασμούς \( i, j, k \):}

$\sum_{k=1}^{n} \sum_{i=1}^{n} \sum_{j=1}^{n} O(1) = O(n) \cdot O(n) \cdot O(n) = O(n^3)$

\subsection*{\textgreek{Μετά από την επεξεργασία μέσω του τριπλού βρόγχου επανάληψης: }}

\textgreek{\textgreek{Έλεγχος διαγώνιων στοιχείων για την εύρεση συντομότερου κύκλου: }}
\[
\min_{v \in V} \text{dist}[v][v]
\]
\textgreek{Αυτό απαιτεί: }  $\sum_{v=1}^{n} O(1) = O(n)$  \textgreek{ πράξεις.}

\subsection*{\textgreek{Συνολική πολυπλοκότητα: }}

$T(n) = \sum_{k=1}^{n} \sum_{i=1}^{n} \sum_{j=1}^{n} O(1) + \sum_{v=1}^{n} O(1) = O(n^3) 
+ O(n) = \mathcal{O}(n^3)$



\section*{\textgreek{Απόδειξη Ορθότητας}}

\textgreek{Έστω $V = \{1, 2, \dots, n\}$ το σύνολο των κόμβων. Ορίζουμε τη συντομότερη διαδρομή από $i$ σε $j$ χρησιμοποιώντας μόνο ενδιάμεσους κόμβους από το υποσύνολο $\{1, 2, \dots, k\}$. Την ονομάζουμε $\delta_k(i,j)$. Η επαγωγική υπόθεση ορίζει ότι για κάθε $k$, ισχύει η εξής σχέση:}

\[
\delta_k(i,j) = \min \left\{ \delta_{k-1}(i,j),\ \delta_{k-1}(i,k) + \delta_{k-1}(k,j) \right\}
\]
\textgreek{
\begin{itemize}
    \item Για $k = 0$ (αν δεν υπάρχει ενδιάμεσος κόμβος ανάμεσα στους κόμβους $i$ και $j$), η $\delta_0(i,j)$ είναι:
    \[
    \delta_0(i,j) =
    \begin{cases}
    w(i,j), & \text{αν υπάρχει ακμή } (i,j) \in A \\
    \infty, & \text{διαφορετικά}
    \end{cases}
    \]
    \item Έστω ότι η πρόταση ισχύει για $k-1$, και εξετάζουμε την περίπτωση $k$:
    \begin{itemize}
        \item Αν η συντομότερη διαδρομή από $i$ σε $j$ **δεν περιλαμβάνει τον $k$** ως ενδιάμεσο κόμβο, τότε όλοι οι ενδιάμεσοι κόμβοι ανήκουν στο $\{1, \dots, k-1\}$ και:
        \[
        \delta_k(i,j) = \delta_{k-1}(i,j)
        \]
        \item Αν η διαδρομή περιλαμβάνει τον $k$ ως ενδιάμεσο, τότε διασπάται σε δύο διαδρομές:
        \[
        \delta_k(i,j) = \delta_{k-1}(i,k) + \delta_{k-1}(k,j)
        \]
        που σύμφωνα με την επαγωγική υπόθεση είναι σωστές.
    \end{itemize}
\end{itemize}
}
\textgreek{
Επομένως, η αναδρομική σχέση υπολογίζει σωστά το $\delta_k(i,j)$ για κάθε $k$, και τελικά το $\delta_n(i,j)$ είναι η συντομότερη διαδρομή από $i$ σε $j$.}

\subsection*{\textgreek{Εύρεση Κύκλου}}
\textgreek{
Η συντομότερη κατευθυνόμενη κυκλική διαδρομή είναι μια διαδρομή που ξεκινά και καταλήγει στον ίδιο κόμβο. Άρα για κάθε $v \in V$ η ποσότητα} $\text{dist}[v][v]$ \textgreek{περιέχει το μήκος της συντομότερης κυκλικής διαδρομής που περνά από το $v$.}

\textgreek{Το ελάχιστο αυτών:}
\[
\min_{v \in V} \text{dist}[v][v]
\]
\textgreek{είναι το μήκος του συντομότερου κατευθυνόμενου κύκλου.}

\section*{Exercise 2}

\begin{algorithm}[H]
\caption{\begin{otherlanguage}{greek}Βρες την Ειδική Κορυφή σε $O(n)$\end{otherlanguage}}
\begin{algorithmic}[1]
\State \textbf{\begin{otherlanguage}{greek}Είσοδος:\end{otherlanguage}} \begin{otherlanguage}{greek}Γράφημα $G$ (ως \end{otherlanguage} adjacency matrix \begin{otherlanguage}{greek}) με $n$ κορυφές και συνάρτηση \texttt{γνωρίζει$(u, v)$} που επιστρέφει $True$ αν υπάρχει ακμή από $u$ προς $v$.\end{otherlanguage}
\State \textbf{\begin{otherlanguage}{greek}Έξοδος:\end{otherlanguage}} \begin{otherlanguage}{greek} $True$ αν υπάρχει ειδική κορυφή, αλλιώς $False$.\end{otherlanguage}
\State

\State \textbf{\begin{otherlanguage}{greek}Βήμα 1: Επιλογή υποψηφίου\end{otherlanguage}}
\State \begin{otherlanguage}{greek}υποψήφιος $\gets 1$\end{otherlanguage}
\For{$i = 2$ \textbf{\begin{otherlanguage}{greek}μέχρι\end{otherlanguage}} $n$}
    \If{\begin{otherlanguage}{greek}\texttt{γνωρίζει(υποψήφιος, $i$)}\end{otherlanguage}}
        \State \begin{otherlanguage}{greek}υποψήφιος $\gets i$\end{otherlanguage}
    \EndIf
\EndFor

\State

\State \textbf{\begin{otherlanguage}{greek}Βήμα 2: Έλεγχος εγκυρότητας\end{otherlanguage}}
\For{$i = 1$ \textbf{\begin{otherlanguage}{greek}μέχρι\end{otherlanguage}} $n$}
    \If{\begin{otherlanguage}{greek}$i$ = υποψήφιος\end{otherlanguage}}
        \State \begin{otherlanguage}{greek}\textbf{συνέχισε}\end{otherlanguage}
    \EndIf
    \If{\begin{otherlanguage}{greek}\textbf{όχι} \texttt{γνωρίζει($i$, υποψήφιος)} \textbf{ή} \texttt{γνωρίζει(υποψήφιος, $i$)}\end{otherlanguage}}
        \State \Return \texttt{False}
    \EndIf
\EndFor

\State \Return \texttt{True}
\end{algorithmic}
\end{algorithm}


\begin{otherlanguage}{greek}
Ο αλγόριθμος εντοπίζει αν υπάρχει μία ειδική κορυφή $c$ στο γράφημα με τις εξής ιδιότητες:
\begin{itemize}
    \item Για κάθε $i \neq c$, ισχύει \texttt{γνωρίζει$(i, c)$} = \texttt{$True$}
    \item Για κάθε $i \neq c$, ισχύει \texttt{γνωρίζει$(c, i)$} = \texttt{$False$}
\end{itemize}

Ο αλγόριθμος αποτελείται από δύο φάσεις:
\end{otherlanguage}

\paragraph{\begin{otherlanguage}{greek}Βήμα 1: Επιλογή Υποψήφιου\end{otherlanguage}}

\begin{otherlanguage}{greek}
Αρχικά θεωρούμε τον κόμβο $1$ ως υποψήφιο. Για κάθε επόμενο κόμβο $i$ από $2$ έως $n$, ελέγχουμε:

\begin{itemize}
    \item Αν ο τρέχων υποψήφιος \texttt{γνωρίζει($i$)}, τότε ο υποψήφιος απορρίπτεται (δεν είναι η ειδική κορυφή), και το $i$ γίνεται ο νέος υποψήφιος.
    \item Αλλιώς, το $i$ δεν μπορεί να είναι η ειδική κορυφή γιατί υπάρχει κάποιος που δεν τον γνωρίζει, και κρατάμε τον ίδιο υποψήφιο.
\end{itemize}

Στο τέλος του πρώτου βήματος, μένει ένας μοναδικός υποψήφιος που ενδέχεται να είναι η ειδική κορυφή.
\end{otherlanguage}

\paragraph{\begin{otherlanguage}{greek}Βήμα 2: Έλεγχος Εγκυρότητας\end{otherlanguage}}

\begin{otherlanguage}{greek}
Για κάθε κόμβο $i \neq$ υποψήφιος, ελέγχουμε:
\begin{itemize}
    \item Αν \texttt{γνωρίζει($i$, υποψήφιος)} είναι \texttt{$False$}, τότε απορρίπτεται.
    \item Αν \texttt{γνωρίζει(υποψήφιος, $i$)} είναι \texttt{$True$}, τότε επίσης απορρίπτεται.
\end{itemize}

\paragraph{Συμπέρασμα:} Αν περάσει τον έλεγχο, επιστρέφεται \texttt{$True$}, αλλιώς \texttt{$False$}.
\end{otherlanguage}

\paragraph{\begin{otherlanguage}{greek}Χρονική Πολυπλοκότητα:\end{otherlanguage}} $O(n)$ \begin{otherlanguage}{greek} για τον 1ο βρόχο επανάληψης και $Ο(n)$ και στο 2ο βρόχο επανάληψης άρα στο σύνολο έχουμε $O(n) + O(n) = O(n)$ , άρα αλγοριθμικά βέλτιστος.\end{otherlanguage}

\section*{Exercise 3}

\begin{algorithm}[H]
\caption{Find Path Between Two Vertices in MST (BFS)}
\begin{algorithmic}[1]
\Function{find\_path}{start, end}
    \State \textgreek{Αρχικοποίηση ουράς} \( q \gets \{start\} \)
    \State \( \text{visited[start]} \gets \text{True} \)
    \While{\( q \) is not empty}
        \State \( \text{node} \gets q.pop() \)
        \If{\( \text{node} == \text{end} \)}
           \Comment{ \textgreek{Ανακατασκευή του μονοπατιού χρησιμοποιώντας τον πίνακα}} \textnormal{parent}
            \State \( \text{curr} \gets \text{end} \)
            \While{\( \text{curr} \ne \text{start} \)}
                \State \( \text{prev} \gets \text{parent[curr]} \)
                \State \( \text{curr} \gets \text{prev} \)
            \EndWhile
            \State \Return \text{True}
        \EndIf
        \ForAll{neighbor \( \in \text{adj\_list[node]} \)}
            \If{not \( \text{visited[neighbor]} \)}
                \State \( \text{visited[neighbor]} \gets \text{True} \)
                \State \( \text{parent[neighbor]} \gets \text{node} \)
                \State \( q.push(neighbor) \)
            \EndIf
        \EndFor
    \EndWhile
    \State \Return False
\EndFunction
\end{algorithmic}
\end{algorithm}

\begin{algorithm}[H]
\caption{handle\_weight\_increase}
\begin{algorithmic}[1]
\State \textgreek{Είσοδος: } \( V, \text{graph\_edges}, \text{mst\_edges}, \text{edge\_to\_remove} \)
\State \textgreek{Έξοδος: } \textgreek{Λίστα με ακμές ενημερωμένου ΕΕΔ}
\State \text{adj\_list} \(\gets\) \textgreek{Άδεια λίστα γειτνίασης μεγέθους \( V \)} 
\For{each edge \( e \in \text{mst\_edges} \)}
    \If{not \((e.u == \text{edge\_to\_remove}.u \text{ and } e.v == \text{edge\_to\_remove}.v)\) or \((e.u == \text{edge\_to\_remove}.v \text{ and } e.v == \text{edge\_to\_remove}.u)\))}
        \State \text{adj\_list[e.u].append(e.v)}
        \State \text{adj\_list[e.v].append(e.u)}
    \EndIf
\EndFor

\State \text{visited} \(\gets\) Array of False values for \( V \)
\State \text{component} \(\gets\) Array of 0 values for \( V \)
\State \text{comp\_id} \(\gets\) 1

\For{i = 0 to V-1}
    \If{visited[i] is False}
        \State \text{FIND\_PATH(i, comp\_id)}  \Comment{\textgreek{\(bfs\) για να βρούμε τις συνιστώσες}}
        \State comp\_id \(\gets\) comp\_id + 1
    \EndIf
\EndFor

\If{\( \text{component[edge\_to\_remove.u]} == \text{component[edge\_to\_remove.v]} \)}
    \State \textbf{return} \text{mst\_edges}
\EndIf

\State \text{min\_edge} \(\gets\) None
\For{each edge \( e \in \text{graph\_edges} \)}
    \If{\( \text{component[e.u]} \neq \text{component[e.v]} \)}
        \If{min\_edge is None or \( e.\text{weight} < \text{min\_edge.weight} \)}
            \State \( \text{min\_edge} \gets e \)
        \EndIf
    \EndIf
\EndFor

\State \text{result} \(\gets\) List of edges in \text{mst\_edges} excluding \text{edge\_to\_remove}
\If{min\_edge is not None}
    \State \text{result.append(min\_edge)}
\EndIf

\State \textbf{return} result
\end{algorithmic}
\end{algorithm}

\begin{algorithm}[H]
\caption{DFS}
\begin{algorithmic}[1]
\Procedure{DFS}{$node$, $target$, $adjList$, $visited$, $parent$, $edgeToParent$}
    \If{$node = target$}
        \State \Return \textbf{true}
    \EndIf
    \State $visited[node] \gets true$
    \ForAll{$(neighbor, edge)$ in $adjList[node]$}
        \If{\textbf{not} $visited[neighbor]$}
            \State $parent[neighbor] \gets node$
            \State $edgeToParent[neighbor] \gets edge$
            \If{\Call{DFS}{$neighbor$, $target$, $adjList$, $visited$, $parent$, $edgeToParent$}}
                \State \Return \textbf{true}
            \EndIf
        \EndIf
    \EndFor
    \State \Return \textbf{false}
\EndProcedure
\end{algorithmic}
\end{algorithm}

\vspace{1em}

\begin{algorithm}[H]
\caption{handle\_weight\_decrease($V$, $graphEdges$, $mstEdges$, $u$, $v$, $newWeight$)}
\begin{algorithmic}[1]
\State \textgreek{Αρχικοποίησε λίστα γειτνίασης} $adjList[0 \dots V-1] \gets \textgreek{Άδειες λίστες}$
\ForAll{$e \in mstEdges$}
    \State Add $(e.v, e)$ to $adjList[e.u]$ and $(e.u, e)$ to $adjList[e.v]$
\EndFor

\State \textgreek{Αρχικοποίησε πίνακες}: $visited[0 \dots V-1] \gets false$, $parent[0 \dots V-1] \gets -1$, $edgeToParent[0 \dots V-1] \gets \text{None}$

\If{\textbf{not} \Call{DFS}{$u$, $v$, $adjList$, $visited$, $parent$, $edgeToParent$}}
    \State \Return $mstEdges$
\EndIf

\State $curr \gets v$, $maxEdge \gets$ None
\While{$curr \ne u$}
    \State $edge \gets edgeToParent[curr]$
    \If{$maxEdge = \text{None}$ \textbf{or} $edge.weight > maxEdge.weight$}
        \State $maxEdge \gets edge$
    \EndIf
    \State $curr \gets parent[curr]$
\EndWhile

\If{$maxEdge \ne \text{None}$ \textbf{and} $maxEdge.weight > newWeight$}
    \State $newMST \gets$ all edges in $mstEdges$ except $maxEdge$
    \State Add edge $(u, v, newWeight)$ to $newMST$
    \State \Return $newMST$
\EndIf

\State \Return $mstEdges$
\end{algorithmic}
\end{algorithm}


\begin{algorithm}[H]
\caption{Update MST }
\begin{algorithmic}[1]
\State \textgreek{Είσοδος: }
\State \quad \( V \) (\textgreek{Αριθμός Κορυφών})
\State \quad \( \text{graph\_edges} \) \textgreek{Λίστα ακμών με τα βάρη τους}
\State \quad \( \text{mst\_edges} \) \textgreek{Λίστα ακμών στο παρών ΕΕΔ}
\State \quad \( u, v \) \textgreek{Κορυφές της ακμής που ενημερώνεται}
\State \quad \( \text{new\_weight} \) \textgreek{ (Νέο βάρος της ακμής \( (u, v) \))}
\State \textgreek{Έξοδος: } \textgreek{Λίστα με τις ακμές του ενημερωμένου ΕΕΔ}

\State \( \text{old\_weight} \gets \text{None} \)
\For{each edge \( e \in \text{graph\_edges} \)}
    \If{(e.u == u and e.v == v) or (e.u == v and e.v == u)}
        \State \( \text{old\_weight} \gets e.\text{weight} \)
        \State \( e.\text{weight} \gets \text{new\_weight} \)
        \State \textbf{break}
    \EndIf
\EndFor

\If{\( \text{old\_weight} \) is \( \text{None} \)}
    \State \text{return None}  \Comment{\textgreek{Η ενημερωμένη ακμή δεν βρέθηκε στον γράφο}}
\EndIf

\State \( \text{edge\_in\_mst} \gets \text{None} \)
\For{each edge \( e \in \text{mst\_edges} \)}
    \If{(e.u == u and e.v == v) or (e.u == v and e.v == u)}
        \State \( \text{edge\_in\_mst} \gets e \)
        \State \( e.\text{weight} \gets \text{new\_weight} \)
        \State \textbf{break}
    \EndIf
\EndFor

\If{edge\_in\_mst is not None and new\_weight \(>\) old\_weight}
    \State \textbf{return} \texttt{handle\_weight\_increase}(V, graph\_edges, mst\_edges, edge\_in\_mst)
\EndIf

\If{edge\_in\_mst is None and new\_weight \(<\) old\_weight}
    \State \textbf{return} \texttt{handle\_weight\_decrease}(V, graph\_edges, mst\_edges, u, v, new\_weight)
\EndIf

\State \textbf{return} mst\_edges
\end{algorithmic}
\end{algorithm}





\section*{Exercise 4}

\begin{algorithm}[H]
\caption{\begin{otherlanguage}{greek}Χρωματισμός Διαστημάτων με Ελάχιστα Χρώματα \end{otherlanguage}}
\begin{algorithmic}[1]
\State \begin{otherlanguage}{greek}Είσοδος: Πίνακας $Set = \{(l_i, r_i)\}_{i=1}^n$ από $n$ ανοιχτά διαστήματα.\end{otherlanguage}
\State \begin{otherlanguage}{greek}Ταξινόμησε τα διαστήματα κατά αύξον $l_i$.\end{otherlanguage}
\State \begin{otherlanguage}{greek}Αρχικοποίησε πίνακα $ColourArray[1 \ldots n] \gets 0$\end{otherlanguage}
\State \begin{otherlanguage}{greek}Αρχικοποίησε κενό $set$ για διαθέσιμα χρώαματα $AvailableColours$\end{otherlanguage}
\State \begin{otherlanguage}{greek}Αρχικοποίησε $CurrentColour \gets 1$\end{otherlanguage}
\State \begin{otherlanguage}{greek}Αρχικοποίησε κενή \end{otherlanguage} priority queue $PQ$ \begin{otherlanguage}{greek} με στοιχεία της μορφής $(r_i, \text{χρώμα})$ ταξινομημένα κατά $r_i$\end{otherlanguage}
\For{$i \gets 1$ \textbf{to} $n$}
    \While{\begin{otherlanguage}{greek}$PQ$ δεν είναι κενή \textbf{και} $PQ.top().first \leq l_i$\end{otherlanguage}}
        \State \begin{otherlanguage}{greek}Αφαίρεσε ($r$, χρώμα) από $PQ$\end{otherlanguage}
        \State \begin{otherlanguage}{greek}Πρόσθεσε \textbf{χρώμα} στο $AvailableColours$\end{otherlanguage}
    \EndWhile
    \If{\begin{otherlanguage}{greek}υπάρχει διαθέσιμο χρώμα\end{otherlanguage}}
        \State \begin{otherlanguage}{greek}Ανάθεσε στον πίνακα $ColourArray[i]$ ένα διαθέσιμο χρώμα από το $AvailableColours$\end{otherlanguage}
    \Else
        \State \begin{otherlanguage}{greek}$ColourArray[i] \gets CurrentColour$\end{otherlanguage}
        \State \begin{otherlanguage}{greek}$CurrentColour \gets CurrentColour + 1$\end{otherlanguage}
    \EndIf
    \State \begin{otherlanguage}{greek}Προσθήκη $(r_i, ColourArray[i])$ στο $PQ$\end{otherlanguage}
\EndFor
\State \Return \begin{otherlanguage}{greek}$ColourArray[1 \ldots n]$\end{otherlanguage}
\end{algorithmic}
\end{algorithm}


\subsection*{\begin{otherlanguage}{greek}Απόδειξη Ορθότητας και Πολυπλοκότητας Αλγορίθμου Χρωματισμού\end{otherlanguage}}

\paragraph{\begin{otherlanguage}{greek}Ιδιότητες που πρέπει να αποδείξουμε:\end{otherlanguage}}

\begin{otherlanguage}{greek}
\begin{itemize}
    \item[(1)] Ο αλγόριθμος δεν αναθέτει ποτέ το ίδιο χρώμα σε δύο επικαλυπτόμενα διαστήματα.
    \item[(2)] Ο αριθμός χρωμάτων που χρησιμοποιούνται είναι ελάχιστος δυνατός.
\end{itemize}
\end{otherlanguage}

\paragraph{\begin{otherlanguage}{greek}Απόδειξη ιδιότητας (1):\end{otherlanguage}}\begin{otherlanguage}{greek} Κανένα δύο επικαλυπτόμενα διαστήματα δεν έχουν το ίδιο χρώμα.\end{otherlanguage}

\begin{otherlanguage}{greek}
Έστω ότι το διάστημα $I_i = (l_i, r_i)$ παίρνει ένα χρώμα $c$ από το σύνολο των διαθέσιμων χρωμάτων $AvailableColours$.

Αυτό σημαίνει ότι το χρώμα $c$ είχε ανατεθεί σε κάποιο προηγούμενο διάστημα $I_j = (l_j, r_j)$, το οποίο όμως είχε ήδη αφαιρεθεί από την ουρά $PQ$ (άρα $r_j \leq l_i$), δηλαδή το $I_j$ έχει τελειώσει πριν ξεκινήσει το $I_i$.

Επομένως, τα $I_j$ και $I_i$ δεν επικαλύπτονται, άρα είναι ασφαλές να χρησιμοποιήσουμε ξανά το ίδιο χρώμα $c$.

Ομοίως, αν το $I_i$ λάβει νέο χρώμα $c$ (δεν υπάρχει διαθέσιμο), αυτό σημαίνει ότι επικαλύπτεται με όλα τα ενεργά διαστήματα που βρίσκονται στην $PQ$, τα οποία έχουν διαφορετικά χρώματα.

Άρα, κανένα δύο επικαλυπτόμενα διαστήματα δεν έχουν το ίδιο χρώμα.
\end{otherlanguage}

\paragraph{\begin{otherlanguage}{greek}Απόδειξη ιδιότητας (2):\end{otherlanguage} } \begin{otherlanguage}{greek} Ο αριθμός χρωμάτων είναι ελάχιστος.\end{otherlanguage}

\begin{otherlanguage}{greek}
Το πλήθος των χρωμάτων που χρησιμοποιούνται αντιστοιχεί στον μέγιστο αριθμό επικαλυπτόμενων διαστημάτων σε οποιοδήποτε χρονικό σημείο. Ο λόγος είναι ότι το χρώμα επαναχρησιμοποιείται αμέσως μόλις το αντίστοιχο διάστημα τελειώσει.

Δηλαδή, ένα νέο χρώμα δημιουργείται μόνο όταν το τρέχον διάστημα επικαλύπτεται με όλα τα προηγούμενα "ενεργά" διαστήματα, και δεν υπάρχει κανένα χρώμα διαθέσιμο.

\end{otherlanguage}

\paragraph{\begin{otherlanguage}{greek}Συμπέρασμα:\end{otherlanguage}}

\begin{otherlanguage}{greek}
Ο αλγόριθμος αναθέτει έγκυρα χρώματα χωρίς επικαλύψεις, και χρησιμοποιεί τον ελάχιστο δυνατό αριθμό χρωμάτων για τη χρωμάτιση των διαστημάτων.
\end{otherlanguage}


\paragraph{\begin{otherlanguage}{greek}Πολυπλοκότητα:\end{otherlanguage}}

\begin{otherlanguage}{greek}
Αναλύουμε τα βήματα του αλγορίθμου:

\begin{itemize}
    \item Ταξινόμηση των $n$ διαστημάτων κατά $l_i$: $O(n \log n)$
    \item Για κάθε διάστημα( $O(n)$ ):
    \begin{itemize}
        \item Εισαγωγή ή αφαίρεση από το \begin{otherlanguage}{english}priority queue: \end{otherlanguage} $O(\log n)$
        \item Αναζήτηση/εκχώρηση χρώματος: $O(1)$ με διαχείριση διαθέσιμων χρωμάτων ($set$).
    \end{itemize}
\end{itemize}

Άρα $O(nlogn)$

\textbf{Συνολικά:} $O(n \log n) + O(n \log n) = O(n \log n)$ λόγω της ταξινόμησης και των προσθηκών/αφαιρέσεων στη ουρά προτεραιότητας ανά διάστημα.

\paragraph{Βέλτιστοτητα:} Ο αλγόριθμος είναι βέλτιστος για τα διαστηματά μας, καθώς ο ελάχιστος αριθμός χρωμάτων είναι ίσος με τη μέγιστη ταυτόχρονη επικάλυψη, κάτι που επιτυγχάνεται εδώ.
\end{otherlanguage}



\end{document}
